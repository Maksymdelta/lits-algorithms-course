\documentclass[12pt,a4paper]{article}

\usepackage{tikz}

% ----------------------------
% Enable multilingual content
% ----------------------------
\usepackage[T1, T2A]{fontenc}
\usepackage[utf8]{inputenc}
\usepackage[ukrainian, english]{babel}

\usepackage{bbding}
\usepackage{ifthen}
\usepackage{pgffor}

\newcommand{\fivestarrating}[1]{%
    \ifthenelse{\not\equal{#1}{0}}{\foreach \n in {1,...,#1}{\FiveStar}}{}%
    \ifthenelse{\not\equal{#1}{5}}{\foreach \n in {\numexpr(#1+1),...,5}{\FiveStarOpen}}{}%
}%

% ---------------------
% Page layout settings
% ---------------------

% Set showframe to true to display layout guides.
\usepackage[showframe=false]{geometry}

% Enable this and use \layout in the document to include a special page with document layout metrics.
%\usepackage{layout}

% Set page margins
\geometry{margin=1in}

% ---------------------------
% Custom formatting for code
% ---------------------------

\usepackage{listings}
\usepackage{color}
\usepackage{inconsolata}
\usepackage{upquote}

\definecolor{commentcolor}{rgb}{0.6,0.6,0.6}


% Style to be used by all Python source listings.
\lstdefinestyle{codePythonStyle}{
    language=Python,
    commentstyle=\color{commentcolor},
    basewidth=0.53em,
    stepnumber=1,
    numbersep=8pt,
    tabsize=4,
    showspaces=false,
    showstringspaces=false
}

% A shorthand syntax for language-neutral inline code blocks, e.g. |foo|
\lstMakeShortInline[language=]|

% A shorthand syntax for language-neutral inline code blocks, e.g. \code{foo}
\newcommand{\code}{\lstinline[language=]}

% Language-neutral multiline code blocks.
\lstnewenvironment{codeblock}
    {\lstset{language=}}
    {}


% Use Python by default.
\lstset{
    basicstyle=\fontencoding{T1}\selectfont\ttfamily\small,
    style=codePythonStyle
}

\usepackage[unicode]{hyperref}

% Colors for hyperlinks
\hypersetup{%
  colorlinks=false,            % hyperlinks will be black
  linkbordercolor={0.3 0.4 1}, % hyperlink borders will be blue
  pdfborderstyle={/S/U/W 1}    % border style will be underline of width 1pt
}

% Avoid header in the TOC
\makeatletter
\renewcommand\tableofcontents{%
    \@starttoc{toc}%
}
\makeatother

% Make hyperlinks underlined instead of bounded by a rectangle.
\makeatletter
\Hy@AtBeginDocument{%
  \def\@pdfborder{0 0 1}            % Overrides border definition set with colorlinks=true
  \def\@pdfborderstyle{/S/U/W 1}    % Overrides border style set with colorlinks=true
                                    % Hyperlink border style will be underline of width 1pt
}
\makeatother

% -------------------------------
% Custom formatting for headings
% -------------------------------
\usepackage{titlesec}

% Chapters
\titleformat
    {\chapter}                      % command
    [display]                       % shape
    {\large\bfseries}               % format
    {\vspace{-3em} Модуль \thechapter}            % label
    {0.5ex}                         % sep
    {
        \rule{\textwidth}{1.5pt}
        \vspace{0.5ex}
        \centering
        \mdseries
        \itshape
    }                               % before-code
    [
        \vspace{-1.7ex}
        \rule{\textwidth}{0.3pt}
    ]                               % after-code

% Sections
\titleformat
    {\section}                      % command
    {\normalsize\bfseries}          % format
    {\thesection}                   % label
    {1em}                           % sep
    {}                              % before-code
    []                              % after-code

% Subsections
\titleformat
    {\subsection}                   % command
    {\normalsize\bfseries\itshape}  % format
    {\thesubsection}                % label
    {1em}                           % sep
    {}                              % before-code
    []                              % after-code

% ----------------
% List formatting
% ----------------
\usepackage{paralist}
    \let\itemize\compactitem
    \let\enditemize\endcompactitem
    \let\enumerate\compactenum
    \let\endenumerate\endcompactenum
    \let\description\compactdesc
    \let\enddescription\endcompactdesc
    \pltopsep=\medskipamount
    \plitemsep=1pt
    \plparsep=1pt

% ---------------------------------
% Custom formatting for paragraphs
% ---------------------------------
\setlength{\parskip}{0.5em}
\setlength{\parindent}{0em}

\newenvironment{pagebottomtext}{\begin{center}\par\vspace*{\fill}}{\clearpage\end{center}}


% ---------------------------------------------
% Custom formatting for input/output examples
% ---------------------------------------------

\usepackage{xcolor}


\titleformat
    {\subsubsection}                % command
    [display]                       % shape
    {\normalsize\mdseries}          % format
    {}                              % label
    {0.5ex}                         % sep
    {
        \colorexample
    }                               % before-code

\newcommand{\colorexample}[1]{%
    \colorbox{gray!20}{%
        \parbox{%
            \dimexpr\textwidth-2\fboxsep
        }%
        {#1}
    }%
}



\begin{document}

\section*{Ігровий сервер \hfill \fivestarrating{4}}


\subsection*{Код задачі: \code{GAMSRV}}

Важливим фактором для багатокористувацької онлайн-гри є низька мережева затримка від користувача до сервера.
При цьому, пристрої в Інтернеті спілкуються один з одним, використовуючи мережеві маршрути, які проходять через низку проміжних вузлів-маршрутизаторів. Кожна ланка цього маршруту має власну ненульову затримку.

\begin{center}
    \footnotesize
    \begin{tikzpicture}
        \path (0,0) node[circle,draw] (client1) {Client 1}
              (2,-3) node[circle,draw] (client2) {Client 2}
              (3,0) node[circle,draw] (router1) {Router 1}
              (6,0) node[circle,draw] (router2) {Router 2}
              (9,0) node[circle,draw] (server) {SERVER}
              (12,0) node[circle,draw] (client3) {Client 3};

        \draw[dotted]
            (client1) -- node[above,sloped] {10 ms} (router1)
                      -- node[above,sloped] {80 ms} (router2)
                      -- node[above,sloped] {50 ms} (server)
                      -- node[above,sloped] {20 ms} (client3)
            (client2) -- node[above,sloped] {40 ms} (router1)
            (client2) -- node[above,sloped] {100 ms} (router2);
    \end{tikzpicture}
\end{center}

\begin{itemize}
    \item Кожен вузол мережі може виконувати одну з трьох ролей: Client, Router або Server.
    \item Server може бути лише один на всю мережу.
    \item Усі комунікації двосторонні: якщо вузол A може спілкуватися з вузлом B, вузол B може спілкуватися з вузлом A з такою ж затримкою.
    \item Якщо існує кілька шляхів від клієнта до сервера, клієнт гарантовано піде шляхом з найменшою сумарною затримкою (навіть якщо цей шлях пролягає через іншого клієнта).
    \item Усі затримки --- сталі додатні числа.
\end{itemize}

Для прикладу вище, затримки до клієнтів становлять:
\begin{itemize}
    \item Client 1: 10 + 80 + 50 = 140 ms
    \item Client 2: 100 + 50 = 150 ms
    \item Client 3: 20 ms
\end{itemize}

Максимальною затримкою в такому випадку є 150 ms. Однак, якщо ми поміняємо ролями вузли ``Router 2'' і ``Server'', затримки скоротяться до 90 ms, 100 ms і 70 ms відповідно, тоді максимальна затримка буде становити 100 ms.

\begin{center}
    \footnotesize
    \begin{tikzpicture}
        \path (0,0) node[circle,draw] (client1) {Client 1}
              (2,-3) node[circle,draw] (client2) {Client 2}
              (3,0) node[circle,draw] (router1) {Router 1}
              (6,0) node[circle,draw] (router2) {SERVER}
              (9,0) node[circle,draw] (server) {Router 2}
              (12,0) node[circle,draw] (client3) {Client 3};

        \draw[dotted]
            (client1) -- node[above,sloped] {10 ms} (router1)
                      -- node[above,sloped] {80 ms} (router2)
                      -- node[above,sloped] {50 ms} (server)
                      -- node[above,sloped] {20 ms} (client3)
            (client2) -- node[above,sloped] {40 ms} (router1)
            (client2) -- node[above,sloped] {100 ms} (router2);
    \end{tikzpicture}
\end{center}


Ви розробляєте онлайн-гру для користувачів зі всієї країни, і бажаєте розмістити центральний ігровий сервер таким чином, щоб максимальна затримка між сервером і кожним клієнтом була мінімальною. В якості сервера можна вибрати будь-який вузол мережі, який не є клієнтом.

Маючи інформацію про топологію мережі (які вузли з’єднані з якими, і яка затримка кожного з’єднання), знайдіть таке розташування сервера, яке мінімізує найбільше значення затримки до клієнта. Виведіть це значення затримки.


\subsection*{Вхідні дані}

Вхідний файл |gamsrv.in| складається з \(M + 2\) рядків.

\begin{itemize}
    \item Перший рядок містить \(N\) і \(M\) --- кількість вузлів та з’єднань відповідно.

    \(3 \leq N \leq 1000\), \(2 \leq M \leq 1000\)
    \item Другий рядок містить перелік цілих чисел, розділених пробілом --- номери вузлів, які є клієнтами. Усі вузли в мережі нумеруються від 1 до \(N\).
    \item Наступні \(M\) рядків містять трійки натуральних чисел \(startnode\), \(endnode\), \(latency\) --- номер початкового вузла, кінцевого вузла та затримка для кожного з’єднання. \(1 \leq latency \leq 10^9\).
\end{itemize}


\subsection*{Вихідні дані}

Вихідний файл |gamsrv.out| повинен містити одне число --- мінімальне значення найбільшої затримки до клієнта (яке ми отримаємо при оптимальному розташуванні сервера).


\pagebreak


\subsubsection*{Приклад 1}

\textbf{\code{gamsrv.in}}

\begin{codeblock}
6 6
1 2 6
1 3 10
3 4 80
4 5 50
5 6 20
2 3 40
2 4 100
\end{codeblock}

\textbf{\code{gamsrv.out}}

\begin{codeblock}
100
\end{codeblock}


\subsubsection*{Приклад 2}

\textbf{\code{gamsrv.in}}

\begin{codeblock}
9 12
2 4 6
1 2 20
2 3 20
3 6 20
6 9 20
9 8 20
8 7 20
7 4 20
4 1 20
5 2 10
5 4 10
5 6 10
5 8 10
\end{codeblock}

\textbf{\code{gamsrv.out}}

\begin{codeblock}
10
\end{codeblock}


\subsubsection*{Приклад 3}

\textbf{\code{gamsrv.in}}

\begin{codeblock}
3 2
1 3
1 2 50
2 3 1000000000
\end{codeblock}

\textbf{\code{gamsrv.out}}

\begin{codeblock}
1000000000
\end{codeblock}


\end{document}