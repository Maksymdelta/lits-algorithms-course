\documentclass[12pt,a4paper]{article}

% ----------------------------
% Enable multilingual content
% ----------------------------
\usepackage[T1, T2A]{fontenc}
\usepackage[utf8]{inputenc}
\usepackage[ukrainian, english]{babel}

\usepackage{bbding}
\usepackage{ifthen}
\usepackage{pgffor}

\newcommand{\fivestarrating}[1]{%
    \ifthenelse{\not\equal{#1}{0}}{\foreach \n in {1,...,#1}{\FiveStar}}{}%
    \ifthenelse{\not\equal{#1}{5}}{\foreach \n in {\numexpr(#1+1),...,5}{\FiveStarOpen}}{}%
}%

% ---------------------
% Page layout settings
% ---------------------

% Set showframe to true to display layout guides.
\usepackage[showframe=false]{geometry}

% Enable this and use \layout in the document to include a special page with document layout metrics.
%\usepackage{layout}

% Set page margins
\geometry{margin=1in}

% ---------------------------
% Custom formatting for code
% ---------------------------

\usepackage{listings}
\usepackage{color}
\usepackage{inconsolata}
\usepackage{upquote}

\definecolor{commentcolor}{rgb}{0.6,0.6,0.6}


% Style to be used by all Python source listings.
\lstdefinestyle{codePythonStyle}{
    language=Python,
    commentstyle=\color{commentcolor},
    basewidth=0.53em,
    stepnumber=1,
    numbersep=8pt,
    tabsize=4,
    showspaces=false,
    showstringspaces=false
}

% A shorthand syntax for language-neutral inline code blocks, e.g. |foo|
\lstMakeShortInline[language=]|

% A shorthand syntax for language-neutral inline code blocks, e.g. \code{foo}
\newcommand{\code}{\lstinline[language=]}

% Language-neutral multiline code blocks.
\lstnewenvironment{codeblock}
    {\lstset{language=}}
    {}


% Use Python by default.
\lstset{
    basicstyle=\fontencoding{T1}\selectfont\ttfamily\small,
    style=codePythonStyle
}

\usepackage[unicode]{hyperref}

% Colors for hyperlinks
\hypersetup{%
  colorlinks=false,            % hyperlinks will be black
  linkbordercolor={0.3 0.4 1}, % hyperlink borders will be blue
  pdfborderstyle={/S/U/W 1}    % border style will be underline of width 1pt
}

% Avoid header in the TOC
\makeatletter
\renewcommand\tableofcontents{%
    \@starttoc{toc}%
}
\makeatother

% Make hyperlinks underlined instead of bounded by a rectangle.
\makeatletter
\Hy@AtBeginDocument{%
  \def\@pdfborder{0 0 1}            % Overrides border definition set with colorlinks=true
  \def\@pdfborderstyle{/S/U/W 1}    % Overrides border style set with colorlinks=true
                                    % Hyperlink border style will be underline of width 1pt
}
\makeatother

% -------------------------------
% Custom formatting for headings
% -------------------------------
\usepackage{titlesec}

% Chapters
\titleformat
    {\chapter}                      % command
    [display]                       % shape
    {\large\bfseries}               % format
    {\vspace{-3em} Модуль \thechapter}            % label
    {0.5ex}                         % sep
    {
        \rule{\textwidth}{1.5pt}
        \vspace{0.5ex}
        \centering
        \mdseries
        \itshape
    }                               % before-code
    [
        \vspace{-1.7ex}
        \rule{\textwidth}{0.3pt}
    ]                               % after-code

% Sections
\titleformat
    {\section}                      % command
    {\normalsize\bfseries}          % format
    {\thesection}                   % label
    {1em}                           % sep
    {}                              % before-code
    []                              % after-code

% Subsections
\titleformat
    {\subsection}                   % command
    {\normalsize\bfseries\itshape}  % format
    {\thesubsection}                % label
    {1em}                           % sep
    {}                              % before-code
    []                              % after-code

% ----------------
% List formatting
% ----------------
\usepackage{paralist}
    \let\itemize\compactitem
    \let\enditemize\endcompactitem
    \let\enumerate\compactenum
    \let\endenumerate\endcompactenum
    \let\description\compactdesc
    \let\enddescription\endcompactdesc
    \pltopsep=\medskipamount
    \plitemsep=1pt
    \plparsep=1pt

% ---------------------------------
% Custom formatting for paragraphs
% ---------------------------------
\setlength{\parskip}{0.5em}
\setlength{\parindent}{0em}

\newenvironment{pagebottomtext}{\begin{center}\par\vspace*{\fill}}{\clearpage\end{center}}


% ---------------------------------------------
% Custom formatting for input/output examples
% ---------------------------------------------

\usepackage{xcolor}


\titleformat
    {\subsubsection}                % command
    [display]                       % shape
    {\normalsize\mdseries}          % format
    {}                              % label
    {0.5ex}                         % sep
    {
        \colorexample
    }                               % before-code

\newcommand{\colorexample}[1]{%
    \colorbox{gray!20}{%
        \parbox{%
            \dimexpr\textwidth-2\fboxsep
        }%
        {#1}
    }%
}



\begin{document}

\section*{Хом’ячки \hfill \fivestarrating{3}}


\subsection*{Код задачі: \code{HAMSTR}}

Зоомагазин займається продажем хом’ячків. Це мирні домашні істоти, проте, як виявилося, вони мають цікаву харчову поведінку.

Для того, щоб прогодувати хом’ячка, який живе наодинці, потрібно \(H\) пакетів корму на день. Однак, якщо кілька хом’ячків живуть разом, у них прокидається жадібність. У такому випадку кожен хом’ячок з’їдає додатково \(G\) пакетів корму в день за кожного сусіда. Денна норма \(H\) та жадібність \(G\) є індивідуальними для кожного хом’ячка.

Всього в магазині є \(C\) хом’ячків. Ви бажаєте придбати якомога більше, проте у вас є всього \(S\) пакетів їжі на день. Визначте максимальну кількість хом’ячків, яку ви можете прогодувати.


\subsection*{Вхідні дані}

Вхідний файл |hamstr.in| складається з \(C + 2\) рядків.

\begin{itemize}
    \item Перший рядок містить \(S\) --- ваш денний запас їжі. \(0 \leq S \leq 10^9 \).
    \item Другий рядок містить \(C\) --- загальна кількість хом’ячків, яка є в продажу. \(1 \leq C \leq 10^5 \).
    \item Наступні \(C\) рядків містять \(H_i, G_i\) --- цілі числа, розділені пробілом, які означають денну норму та рівень жадібності кожного хом’ячка. \(0 \leq H_i, G_i \leq 10^9 \).
\end{itemize}


\subsection*{Вихідні дані}

Вихідний файл |hamstr.out| повинен містити одне число --- максимальна кількість хом’ячків, яку ви зможете прогодувати, поселивши їх разом.


\begin{pagebottomtext}
$\downarrow$ Див. приклади нижче $\downarrow$
\end{pagebottomtext}


\pagebreak


\subsubsection*{Приклад 1}

\textbf{\code{hamstr.in}}

\begin{codeblock}
7
3
1 2
2 2
3 1
\end{codeblock}

\textbf{\code{hamstr.out}}

\begin{codeblock}
2
\end{codeblock}
\emph{Пояснення:} Можна взяти першого хом’ячка та будь-якого з інших двох.


\subsubsection*{Приклад 2}

\textbf{\code{hamstr.in}}

\begin{codeblock}
19
4
5 0
2 2
1 4
5 1
\end{codeblock}

\textbf{\code{hamstr.out}}

\begin{codeblock}
3
\end{codeblock}
\emph{Пояснення:} Третій хом’ячок надто жадібний. Можна взяти всіх інших трьох, тоді за день вони з’їдять \((5 + 0 \cdot 2) + (2 + 2 \cdot 2) + (5 + 1 \cdot 2) = 18\) пакетів


\subsubsection*{Приклад 3}

\textbf{\code{hamstr.in}}

\begin{codeblock}
2
2
1 50000
1 60000
\end{codeblock}

\textbf{\code{hamstr.out}}

\begin{codeblock}
1
\end{codeblock}
\emph{Пояснення:} Обидва хом’ячки надто жадібні, щоб їсти разом.


\end{document}
