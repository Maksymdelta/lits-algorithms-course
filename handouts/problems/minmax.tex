\section*{(Зразок задачі) \hfill \fivestarrating{1}}


\subsection*{Код задачі: \code{MINMAX}}

Знайдіть у масиві найменше та найбільше число.


\subsection*{Вхідні дані}

Вхідний файл |minmax.in| складається з одного рядка, який містить перелік цілих чисел від 0 до \(10 ^ {15}\) включно, розділених пробілом --- елементи масиву. Загальна кількість елементів може бути від 1 до 1~000~000.


\subsection*{Вихідні дані}

Вихідний файл |minmax.out| повинен містити два числа, розділені пробілом --- значення мінімального та максимального елементів.


\subsubsection*{Приклад 1}

\textbf{\code{minmax.in}}

\begin{codeblock}
50 20 30 47 100
\end{codeblock}

\textbf{\code{minmax.out}}

\begin{codeblock}
20 100
\end{codeblock}


\subsubsection*{Приклад 2}

\textbf{\code{minmax.in}}

\begin{codeblock}
1
\end{codeblock}

\textbf{\code{minmax.out}}

\begin{codeblock}
1 1
\end{codeblock}
