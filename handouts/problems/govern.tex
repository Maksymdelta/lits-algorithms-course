\documentclass[12pt,a4paper]{article}

% ----------------------------
% Enable multilingual content
% ----------------------------
\usepackage[T1, T2A]{fontenc}
\usepackage[utf8]{inputenc}
\usepackage[ukrainian, english]{babel}

\usepackage{bbding}
\usepackage{ifthen}
\usepackage{pgffor}

\newcommand{\fivestarrating}[1]{%
    \ifthenelse{\not\equal{#1}{0}}{\foreach \n in {1,...,#1}{\FiveStar}}{}%
    \ifthenelse{\not\equal{#1}{5}}{\foreach \n in {\numexpr(#1+1),...,5}{\FiveStarOpen}}{}%
}%

% ---------------------
% Page layout settings
% ---------------------

% Set showframe to true to display layout guides.
\usepackage[showframe=false]{geometry}

% Enable this and use \layout in the document to include a special page with document layout metrics.
%\usepackage{layout}

% Set page margins
\geometry{margin=1in}

% ---------------------------
% Custom formatting for code
% ---------------------------

\usepackage{listings}
\usepackage{color}
\usepackage{inconsolata}
\usepackage{upquote}

\definecolor{commentcolor}{rgb}{0.6,0.6,0.6}


% Style to be used by all Python source listings.
\lstdefinestyle{codePythonStyle}{
    language=Python,
    commentstyle=\color{commentcolor},
    basewidth=0.53em,
    stepnumber=1,
    numbersep=8pt,
    tabsize=4,
    showspaces=false,
    showstringspaces=false
}

% A shorthand syntax for language-neutral inline code blocks, e.g. |foo|
\lstMakeShortInline[language=]|

% A shorthand syntax for language-neutral inline code blocks, e.g. \code{foo}
\newcommand{\code}{\lstinline[language=]}

% Language-neutral multiline code blocks.
\lstnewenvironment{codeblock}
    {\lstset{language=}}
    {}


% Use Python by default.
\lstset{
    basicstyle=\fontencoding{T1}\selectfont\ttfamily\small,
    style=codePythonStyle
}

\usepackage[unicode]{hyperref}

% Colors for hyperlinks
\hypersetup{%
  colorlinks=false,            % hyperlinks will be black
  linkbordercolor={0.3 0.4 1}, % hyperlink borders will be blue
  pdfborderstyle={/S/U/W 1}    % border style will be underline of width 1pt
}

% Avoid header in the TOC
\makeatletter
\renewcommand\tableofcontents{%
    \@starttoc{toc}%
}
\makeatother

% Make hyperlinks underlined instead of bounded by a rectangle.
\makeatletter
\Hy@AtBeginDocument{%
  \def\@pdfborder{0 0 1}            % Overrides border definition set with colorlinks=true
  \def\@pdfborderstyle{/S/U/W 1}    % Overrides border style set with colorlinks=true
                                    % Hyperlink border style will be underline of width 1pt
}
\makeatother

% -------------------------------
% Custom formatting for headings
% -------------------------------
\usepackage{titlesec}

% Chapters
\titleformat
    {\chapter}                      % command
    [display]                       % shape
    {\large\bfseries}               % format
    {\vspace{-3em} Модуль \thechapter}            % label
    {0.5ex}                         % sep
    {
        \rule{\textwidth}{1.5pt}
        \vspace{0.5ex}
        \centering
        \mdseries
        \itshape
    }                               % before-code
    [
        \vspace{-1.7ex}
        \rule{\textwidth}{0.3pt}
    ]                               % after-code

% Sections
\titleformat
    {\section}                      % command
    {\normalsize\bfseries}          % format
    {\thesection}                   % label
    {1em}                           % sep
    {}                              % before-code
    []                              % after-code

% Subsections
\titleformat
    {\subsection}                   % command
    {\normalsize\bfseries\itshape}  % format
    {\thesubsection}                % label
    {1em}                           % sep
    {}                              % before-code
    []                              % after-code

% ----------------
% List formatting
% ----------------
\usepackage{paralist}
    \let\itemize\compactitem
    \let\enditemize\endcompactitem
    \let\enumerate\compactenum
    \let\endenumerate\endcompactenum
    \let\description\compactdesc
    \let\enddescription\endcompactdesc
    \pltopsep=\medskipamount
    \plitemsep=1pt
    \plparsep=1pt

% ---------------------------------
% Custom formatting for paragraphs
% ---------------------------------
\setlength{\parskip}{0.5em}
\setlength{\parindent}{0em}

\newenvironment{pagebottomtext}{\begin{center}\par\vspace*{\fill}}{\clearpage\end{center}}


% ---------------------------------------------
% Custom formatting for input/output examples
% ---------------------------------------------

\usepackage{xcolor}


\titleformat
    {\subsubsection}                % command
    [display]                       % shape
    {\normalsize\mdseries}          % format
    {}                              % label
    {0.5ex}                         % sep
    {
        \colorexample
    }                               % before-code

\newcommand{\colorexample}[1]{%
    \colorbox{gray!20}{%
        \parbox{%
            \dimexpr\textwidth-2\fboxsep
        }%
        {#1}
    }%
}



\begin{document}

\section*{Державна установа \hfill \fivestarrating{3}}


\subsection*{Код задачі: \code{GOVERN}}

Для закордонної поїздки вам потрібно отримати кілька довідок від держустанови. Проте, виявилося, що для отримання цих довідок потрібні інші довідки, а ті, в свою чергу, потребують ще інших довідок.

На отримання кожної довідки потрібно вистояти чергу, тому ви хочете зекономити час, не стоячи в черзі дарма (якщо ви не будете мати потрібних довідок, стоячи в черзі, вам відмовлять у видачі цієї).

Маючи інформацію про те, які довідки потрібні для яких, визначте оптимальний порядок отримання усіх довідок, при якому вам жодного разу не відмовлять у видачі. Якщо таких оптимальних варіантів кілька --- виведіть будь-який із них.


\subsection*{Вхідні дані}

Кожен із \(N\) рядків вхідного файлу |govern.in| містить два слова, розділені пробілом --- назва довідки та довідка, яку потрібно отримати перед нею.

\begin{itemize}
    \item Рядків може бути від 1 до 100000.
    \item Слова мають довжину від 1 до 50 літер і складаються з малих літер латинського алфавіту від |a| до |z|.
    \item Якщо для однієї довідки потрібно отримати N інших, файл міститиме N рядків, що починатимуться на одне й те саме слово.
    \item Гарантовано існує хоча б один порядок отримання довідок, при якому можна отримати усі довідки.
\end{itemize}

Кодування файлу --- UTF-8 (without BOM).


\subsection*{Вихідні дані}

Вихідний файл |govern.out| повинен містити \(M\) рядків --- назви довідок в порядку їх рекомендованого отримання.

\emph{Примітка:} Після останнього слова також повинен бути символ переносу рядка (newline).


\pagebreak


\subsubsection*{Приклад 1}

\textbf{\code{govern.in}}

\begin{codeblock}
visa foreignpassport
visa hotel
visa bankstatement
bankstatement nationalpassport
hotel creditcard
creditcard nationalpassport
nationalpassport birthcertificate
foreignpassport nationalpassport
foreignpassport militarycertificate
militarycertificate nationalpassport
\end{codeblock}

\textbf{\code{govern.out}}

\begin{codeblock}
birthcertificate
nationalpassport
militarycertificate
foreignpassport
creditcard
hotel
bankstatement
visa
\end{codeblock}


\subsubsection*{Приклад 2}

\textbf{\code{govern.in}}

\begin{codeblock}
visa foreignpassport
\end{codeblock}

\textbf{\code{govern.out}}

\begin{codeblock}
foreignpassport
visa
\end{codeblock}


\subsubsection*{Приклад 3}

\textbf{\code{govern.in}}

\begin{codeblock}
visa
\end{codeblock}

\textbf{\code{govern.out}}

\begin{codeblock}
visa
\end{codeblock}


\end{document}