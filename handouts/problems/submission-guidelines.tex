\documentclass[12pt,a4paper]{article}

% ----------------------------
% Enable multilingual content
% ----------------------------
\usepackage[T1, T2A]{fontenc}
\usepackage[utf8]{inputenc}
\usepackage[ukrainian, english]{babel}

% ---------------------
% Page layout settings
% ---------------------

% Set showframe to true to display layout guides.
\usepackage[showframe=false]{geometry}

% Enable this and use \layout in the document to include a special page with document layout metrics.
%\usepackage{layout}

% Set page margins
\geometry{margin=1in}

% ---------------------------
% Custom formatting for code
% ---------------------------

\usepackage{listings}
\usepackage{color}
\usepackage{inconsolata}
\usepackage{upquote}

\definecolor{commentcolor}{rgb}{0.6,0.6,0.6}


% Style to be used by all Python source listings.
\lstdefinestyle{codePythonStyle}{
    language=Python,
    commentstyle=\color{commentcolor},
    basewidth=0.53em,
    stepnumber=1,
    numbersep=8pt,
    tabsize=4,
    showspaces=false,
    showstringspaces=false
}

% A shorthand syntax for language-neutral inline code blocks, e.g. |foo|
\lstMakeShortInline[language=]|

% A shorthand syntax for language-neutral inline code blocks, e.g. \code{foo}
\newcommand{\code}{\lstinline[language=]}

% Language-neutral multiline code blocks.
\lstnewenvironment{codeblock}
    {\lstset{language=}}
    {}


% Use Python by default.
\lstset{
    basicstyle=\fontencoding{T1}\selectfont\ttfamily\small,
    style=codePythonStyle
}

\usepackage[unicode]{hyperref}

% Colors for hyperlinks
\hypersetup{%
  colorlinks=false,            % hyperlinks will be black
  linkbordercolor={0.3 0.4 1}, % hyperlink borders will be blue
  pdfborderstyle={/S/U/W 1}    % border style will be underline of width 1pt
}

% Avoid header in the TOC
\makeatletter
\renewcommand\tableofcontents{%
    \@starttoc{toc}%
}
\makeatother

% Make hyperlinks underlined instead of bounded by a rectangle.
\makeatletter
\Hy@AtBeginDocument{%
  \def\@pdfborder{0 0 1}            % Overrides border definition set with colorlinks=true
  \def\@pdfborderstyle{/S/U/W 1}    % Overrides border style set with colorlinks=true
                                    % Hyperlink border style will be underline of width 1pt
}
\makeatother

% -------------------------------
% Custom formatting for headings
% -------------------------------
\usepackage{titlesec}

% Chapters
\titleformat
    {\chapter}                      % command
    [display]                       % shape
    {\large\bfseries}               % format
    {\vspace{-3em} Модуль \thechapter}            % label
    {0.5ex}                         % sep
    {
        \rule{\textwidth}{1.5pt}
        \vspace{0.5ex}
        \centering
        \mdseries
        \itshape
    }                               % before-code
    [
        \vspace{-1.7ex}
        \rule{\textwidth}{0.3pt}
    ]                               % after-code

% Sections
\titleformat
    {\section}                      % command
    {\normalsize\bfseries}          % format
    {\thesection}                   % label
    {1em}                           % sep
    {}                              % before-code
    []                              % after-code

% Subsections
\titleformat
    {\subsection}                   % command
    {\normalsize\bfseries\itshape}  % format
    {\thesubsection}                % label
    {1em}                           % sep
    {}                              % before-code
    []                              % after-code

% ----------------
% List formatting
% ----------------
\usepackage{paralist}
    \let\itemize\compactitem
    \let\enditemize\endcompactitem
    \let\enumerate\compactenum
    \let\endenumerate\endcompactenum
    \let\description\compactdesc
    \let\enddescription\endcompactdesc
    \pltopsep=\medskipamount
    \plitemsep=1pt
    \plparsep=1pt

% ---------------------------------
% Custom formatting for paragraphs
% ---------------------------------
\setlength{\parskip}{0.5em}
\setlength{\parindent}{0em}

\newenvironment{pagebottomtext}{\begin{center}\par\vspace*{\fill}}{\clearpage\end{center}}



\begin{document}

\section*{Як робити домашні завдання}


\subsection*{Формат програми}

Розв'язком кожної задачі має бути консольна програма, яка читає дані з вказаного в умові файлу, і записує результати у вказаний в умові файл.

\begin{enumerate}
    \item Не потрібно робити ввід/вивід з клавіатури чи графічний інтерфейс.
    \item Імена файлів вводу/виводу можна «хардкодити» в програмі, проте вказуйте відносні, а не абсолютні шляхи.

        \vspace{0.5em}
        \emph{Корисно:} Щоб легше було відлагоджувати програму і тестувати її на різних наборах даних, ви можете (на власний розсуд) передбачити додаткову функцію: якщо програмі були передані імена файлів в командному рядку, тоді використовувати їх, інакше --- імена за замовчуванням.

        \vspace{0.5em}
        Приклад:
        \begin{itemize}
            \item |"discnt.exe"| --- читатиме дані з |discnt.in| та виводитиме результати в |discnt.out|.
            \item |"discnt.exe case1.in case1.out"| --- читатиме дані з |case1.in| та виводитиме результати в |case1.out|.
        \end{itemize}

    \item Коректність вхідних даних не потрібно перевіряти.
    \item Формат чисел для вхідних/вихідних даних:
        \begin{itemize}
            \item Розділювачів тисяч немає. Наприклад, 5000, 1000000, 123456789000.
            \item Дробовий знак --- крапка. Наприклад, 0.333, 12.75, 9.0.
        \end{itemize}

    \item Кодування вхідних та вихідних файлів --- UTF-8 (without BOM), якщо інше не вказано в умові.
\end{enumerate}


\subsection*{Як здавати рішення}

\begin{enumerate}
    \item Якщо ви ще не створили репозиторію на GitHub, створіть його. Ім’я може бути довільне, наприклад, |lits-ads-002|.
    \item Зверніть увагу на 6-символьний код задачі, вказаний в її описі. Наприклад, |DISCNT|.
    \item Для кожної задачі створюйте окрему папку, наприклад |/lits-ads-002/discnt|. Назва папки обов’язково повинна співпадати з 6-символьною назвою задачі.
    \item В кожній папці задачі зберігайте ваш програмний код.

    Наприклад, |/lits-ads-002/discnt/discnt.py|. Файлів може бути більше ніж один, проте бажано, щоб головна програма називалася так само, як задача. Скомпільованих бінарних файлів зберігати не потрібно (використовуйте файл |.gitignore|: детальніше --- \href{http://git-scm.com/docs/gitignore}{документація}, \href{https://github.com/github/gitignore}{зразки на GitHub})

    \item Ваші рішення будуть автоматично тестуватися кожні 3 години. Слідкуйте за сторінкою \href{http://lits-ads-002.s3-website.eu-central-1.amazonaws.com/}{http://lits-ads-002.s3-website.eu-central-1.amazonaws.com/}, щоб дізнатися про результати.
\end{enumerate}


\end{document}