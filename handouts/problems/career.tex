\documentclass[12pt,a4paper]{article}

\usepackage{tikz}

\usetikzlibrary{arrows.meta}
\usetikzlibrary{shapes.misc}

% ----------------------------
% Enable multilingual content
% ----------------------------
\usepackage[T1, T2A]{fontenc}
\usepackage[utf8]{inputenc}
\usepackage[ukrainian, english]{babel}

\usepackage{bbding}
\usepackage{ifthen}
\usepackage{pgffor}

\newcommand{\fivestarrating}[1]{%
    \ifthenelse{\not\equal{#1}{0}}{\foreach \n in {1,...,#1}{\FiveStar}}{}%
    \ifthenelse{\not\equal{#1}{5}}{\foreach \n in {\numexpr(#1+1),...,5}{\FiveStarOpen}}{}%
}%

% ---------------------
% Page layout settings
% ---------------------

% Set showframe to true to display layout guides.
\usepackage[showframe=false]{geometry}

% Enable this and use \layout in the document to include a special page with document layout metrics.
%\usepackage{layout}

% Set page margins
\geometry{margin=1in}

% ---------------------------
% Custom formatting for code
% ---------------------------

\usepackage{listings}
\usepackage{color}
\usepackage{inconsolata}
\usepackage{upquote}

\definecolor{commentcolor}{rgb}{0.6,0.6,0.6}


% Style to be used by all Python source listings.
\lstdefinestyle{codePythonStyle}{
    language=Python,
    commentstyle=\color{commentcolor},
    basewidth=0.53em,
    stepnumber=1,
    numbersep=8pt,
    tabsize=4,
    showspaces=false,
    showstringspaces=false
}

% A shorthand syntax for language-neutral inline code blocks, e.g. |foo|
\lstMakeShortInline[language=]|

% A shorthand syntax for language-neutral inline code blocks, e.g. \code{foo}
\newcommand{\code}{\lstinline[language=]}

% Language-neutral multiline code blocks.
\lstnewenvironment{codeblock}
    {\lstset{language=}}
    {}


% Use Python by default.
\lstset{
    basicstyle=\fontencoding{T1}\selectfont\ttfamily\small,
    style=codePythonStyle
}

\usepackage[unicode]{hyperref}

% Colors for hyperlinks
\hypersetup{%
  colorlinks=false,            % hyperlinks will be black
  linkbordercolor={0.3 0.4 1}, % hyperlink borders will be blue
  pdfborderstyle={/S/U/W 1}    % border style will be underline of width 1pt
}

% Avoid header in the TOC
\makeatletter
\renewcommand\tableofcontents{%
    \@starttoc{toc}%
}
\makeatother

% Make hyperlinks underlined instead of bounded by a rectangle.
\makeatletter
\Hy@AtBeginDocument{%
  \def\@pdfborder{0 0 1}            % Overrides border definition set with colorlinks=true
  \def\@pdfborderstyle{/S/U/W 1}    % Overrides border style set with colorlinks=true
                                    % Hyperlink border style will be underline of width 1pt
}
\makeatother

% -------------------------------
% Custom formatting for headings
% -------------------------------
\usepackage{titlesec}

% Chapters
\titleformat
    {\chapter}                      % command
    [display]                       % shape
    {\large\bfseries}               % format
    {\vspace{-3em} Модуль \thechapter}            % label
    {0.5ex}                         % sep
    {
        \rule{\textwidth}{1.5pt}
        \vspace{0.5ex}
        \centering
        \mdseries
        \itshape
    }                               % before-code
    [
        \vspace{-1.7ex}
        \rule{\textwidth}{0.3pt}
    ]                               % after-code

% Sections
\titleformat
    {\section}                      % command
    {\normalsize\bfseries}          % format
    {\thesection}                   % label
    {1em}                           % sep
    {}                              % before-code
    []                              % after-code

% Subsections
\titleformat
    {\subsection}                   % command
    {\normalsize\bfseries\itshape}  % format
    {\thesubsection}                % label
    {1em}                           % sep
    {}                              % before-code
    []                              % after-code

% ----------------
% List formatting
% ----------------
\usepackage{paralist}
    \let\itemize\compactitem
    \let\enditemize\endcompactitem
    \let\enumerate\compactenum
    \let\endenumerate\endcompactenum
    \let\description\compactdesc
    \let\enddescription\endcompactdesc
    \pltopsep=\medskipamount
    \plitemsep=1pt
    \plparsep=1pt

% ---------------------------------
% Custom formatting for paragraphs
% ---------------------------------
\setlength{\parskip}{0.5em}
\setlength{\parindent}{0em}

\newenvironment{pagebottomtext}{\begin{center}\par\vspace*{\fill}}{\clearpage\end{center}}


% ---------------------------------------------
% Custom formatting for input/output examples
% ---------------------------------------------

\usepackage{xcolor}


\titleformat
    {\subsubsection}                % command
    [display]                       % shape
    {\normalsize\mdseries}          % format
    {}                              % label
    {0.5ex}                         % sep
    {
        \colorexample
    }                               % before-code

\newcommand{\colorexample}[1]{%
    \colorbox{gray!20}{%
        \parbox{%
            \dimexpr\textwidth-2\fboxsep
        }%
        {#1}
    }%
}


\tikzset{
    basic/.style  = {draw, text width=20pt, circle, align=center}
}

\begin{document}

\section*{Кар’єра \hfill \fivestarrating{2}}


\subsection*{Код задачі: \code{CAREER}}

Ви хочете зробити кар’єру у великій корпорації, яка має складну ієрархічну структуру та багато посад.
Проте, читаючи відгуки працівників на GlassDoor, ви дізнаєтеся, що різні посади в цій компанії приносять зовсім різну кількість корисного досвіду, тому є сенс ретельно обирати, на яких посадах ви хочете працювати.

Організаційна структура компанії має форму піраміди, де вищий рівень має рівно на 1 посаду менше, ніж нижчий.
Досвід, який можна здобути на кожній посаді, а також способи підвищення вказані на схемі:

\begin{center}
    \begin{tikzpicture}
    \begin{scope}[every node/.style={basic}]
    \node                                                  (level1-1) {4};
    \node [below of = level1-1,xshift=-30pt,yshift=-20pt]  (level2-1) {3};
    \node [below of = level1-1,xshift= 30pt,yshift=-20pt]  (level2-2) {1};
    \node [below of = level2-1,xshift=-30pt,yshift=-20pt]  (level3-1) {2};
    \node [below of = level2-1,xshift= 30pt,yshift=-20pt]  (level3-2) {1};
    \node [below of = level2-2,xshift= 30pt,yshift=-20pt]  (level3-3) {5};
    \node [below of = level3-1,xshift=-30pt,yshift=-20pt]  (level4-1) {1};
    \node [below of = level3-1,xshift= 30pt,yshift=-20pt]  (level4-2) {3};
    \node [below of = level3-3,xshift=-30pt,yshift=-20pt]  (level4-3) {2};
    \node [below of = level3-3,xshift= 30pt,yshift=-20pt]  (level4-4) {1};

    \path [draw,->,thick,-{Latex[length=2.5mm]}] (level4-4) -- (level3-3);
    \path [draw,->,thick,-{Latex[length=2.5mm]}] (level4-3) -- (level3-3);
    \path [draw,->,thick,-{Latex[length=2.5mm]}] (level4-3) -- (level3-2);
    \path [draw,->,thick,-{Latex[length=2.5mm]}] (level4-2) -- (level3-2);
    \path [draw,->,thick,-{Latex[length=2.5mm]}] (level4-2) -- (level3-1);
    \path [draw,->,thick,-{Latex[length=2.5mm]}] (level4-1) -- (level3-1);
    \path [draw,->,thick,-{Latex[length=2.5mm]}] (level3-3) -- (level2-2);
    \path [draw,->,thick,-{Latex[length=2.5mm]}] (level3-2) -- (level2-2);
    \path [draw,->,thick,-{Latex[length=2.5mm]}] (level3-2) -- (level2-1);
    \path [draw,->,thick,-{Latex[length=2.5mm]}] (level3-1) -- (level2-1);
    \path [draw,->,thick,-{Latex[length=2.5mm]}] (level2-2) -- (level1-1);
    \path [draw,->,thick,-{Latex[length=2.5mm]}] (level2-1) -- (level1-1);
    \end{scope}

    \end{tikzpicture}
\end{center}

Працівник може бути переведений тільки на вищу посаду (з вищої на нижчу рухатись не дозволяється).

Знаючи досвід, який можна здобути на кожній посаді в компанії, визначте максимальну суму досвіду, яку ви можете здобути, почавши працювати на найнижчому рівні.


\subsection*{Вхідні дані}

Вхідний файл |career.in| складається з \(L + 1\) рядків.

\begin{itemize}
    \item Перший рядок містить \(L\) --- кількість організаційних рівнів в компанії.

          \(1 \leq L \leq 1000\)

    \item Наступні \(L\) рядків містять \(1, 2, 3, ..., L-2, L-1, L\) натуральних чисел \(E\) --- досвід для кожної посади на даному рівні.

          \(1 \leq E < 10000\)
\end{itemize}


\subsection*{Вихідні дані}

Вихідний файл |career.out| повинен містити одне ціле число --- максимальний сумарний досвід, який можливо здобути в цій компанії.


\pagebreak


\subsubsection*{Приклад 1}

\textbf{\code{career.in}}

\begin{codeblock}
4
4
3 1
2 1 5
1 3 2 1
\end{codeblock}

\textbf{\code{career.out}}

\begin{codeblock}
12
\end{codeblock}


\subsubsection*{Приклад 2}

\textbf{\code{career.in}}

\begin{codeblock}
1
9999
\end{codeblock}

\textbf{\code{career.out}}

\begin{codeblock}
9999
\end{codeblock}


\subsubsection*{Приклад 3}

\textbf{\code{career.in}}

\begin{codeblock}
5
0
1 1
0 0 0
1 1 1 1
0 1 0 1 0
\end{codeblock}

\textbf{\code{career.out}}

\begin{codeblock}
3
\end{codeblock}


\end{document}
