\documentclass[12pt,a4paper]{report}

\usepackage{amsmath}
\usepackage{booktabs}
\usepackage{tikz}

\usetikzlibrary{arrows.meta}
\usetikzlibrary{shapes.misc}

% The pipe character gets overriden in the code style definition, specify the table header beforehand.
\newcommand{\adjmatrixlayout}{|r|c|c|c|c|c|c|c|}

% ----------------------------
% Enable multilingual content
% ----------------------------
\usepackage[T1, T2A]{fontenc}
\usepackage[utf8]{inputenc}
\usepackage[ukrainian, english]{babel}

% ---------------------
% Page layout settings
% ---------------------

% Set showframe to true to display layout guides.
\usepackage[showframe=false]{geometry}

% Enable this and use \layout in the document to include a special page with document layout metrics.
%\usepackage{layout}

% Set page margins
\geometry{margin=1in}

% ---------------------------
% Custom formatting for code
% ---------------------------

\usepackage{listings}
\usepackage{color}
\usepackage{inconsolata}
\usepackage{upquote}

\definecolor{commentcolor}{rgb}{0.6,0.6,0.6}


% Style to be used by all Python source listings.
\lstdefinestyle{codePythonStyle}{
    language=Python,
    commentstyle=\color{commentcolor},
    basewidth=0.53em,
    stepnumber=1,
    numbersep=8pt,
    tabsize=4,
    showspaces=false,
    showstringspaces=false
}

% A shorthand syntax for language-neutral inline code blocks, e.g. |foo|
\lstMakeShortInline[language=]|

% A shorthand syntax for language-neutral inline code blocks, e.g. \code{foo}
\newcommand{\code}{\lstinline[language=]}

% Language-neutral multiline code blocks.
\lstnewenvironment{codeblock}
    {\lstset{language=}}
    {}


% Use Python by default.
\lstset{
    basicstyle=\fontencoding{T1}\selectfont\ttfamily\small,
    style=codePythonStyle
}

\usepackage[unicode]{hyperref}

% Colors for hyperlinks
\hypersetup{%
  colorlinks=false,            % hyperlinks will be black
  linkbordercolor={0.3 0.4 1}, % hyperlink borders will be blue
  pdfborderstyle={/S/U/W 1}    % border style will be underline of width 1pt
}

% Avoid header in the TOC
\makeatletter
\renewcommand\tableofcontents{%
    \@starttoc{toc}%
}
\makeatother

% Make hyperlinks underlined instead of bounded by a rectangle.
\makeatletter
\Hy@AtBeginDocument{%
  \def\@pdfborder{0 0 1}            % Overrides border definition set with colorlinks=true
  \def\@pdfborderstyle{/S/U/W 1}    % Overrides border style set with colorlinks=true
                                    % Hyperlink border style will be underline of width 1pt
}
\makeatother

% -------------------------------
% Custom formatting for headings
% -------------------------------
\usepackage{titlesec}

% Chapters
\titleformat
    {\chapter}                      % command
    [display]                       % shape
    {\large\bfseries}               % format
    {\vspace{-3em} Модуль \thechapter}            % label
    {0.5ex}                         % sep
    {
        \rule{\textwidth}{1.5pt}
        \vspace{0.5ex}
        \centering
        \mdseries
        \itshape
    }                               % before-code
    [
        \vspace{-1.7ex}
        \rule{\textwidth}{0.3pt}
    ]                               % after-code

% Sections
\titleformat
    {\section}                      % command
    {\normalsize\bfseries}          % format
    {\thesection}                   % label
    {1em}                           % sep
    {}                              % before-code
    []                              % after-code

% Subsections
\titleformat
    {\subsection}                   % command
    {\normalsize\bfseries\itshape}  % format
    {\thesubsection}                % label
    {1em}                           % sep
    {}                              % before-code
    []                              % after-code

% ----------------
% List formatting
% ----------------
\usepackage{paralist}
    \let\itemize\compactitem
    \let\enditemize\endcompactitem
    \let\enumerate\compactenum
    \let\endenumerate\endcompactenum
    \let\description\compactdesc
    \let\enddescription\endcompactdesc
    \pltopsep=\medskipamount
    \plitemsep=1pt
    \plparsep=1pt

% ---------------------------------
% Custom formatting for paragraphs
% ---------------------------------
\setlength{\parskip}{0.5em}
\setlength{\parindent}{0em}

\newenvironment{pagebottomtext}{\begin{center}\par\vspace*{\fill}}{\clearpage\end{center}}




% TODO: Remove this after merging the modules into a single master file.
\setcounter{chapter}{3}

\tikzset{
    basic/.style = {draw, text width=30pt, circle, align=center}
}

\begin{document}

\chapter{Графи}

% Insert table of contents.
\begingroup
\let\clearpage\relax
\tableofcontents
\endgroup



\section{Термінологія}

Графи дозволяють моделювати складні взаємовідносини між об’єктами. Наприклад, стосунки між користувачами соціальної мережі чи карту автомобільних доріг міста.

З формальної точки зору, граф складається із двох множин --- множини вершин \(V\) (vertices) та ребер \(E\) (edges), що їх з’єднують. Кожне ребро \(e_i\) з’єднує рівно дві вершини \(v_{start}\), \(v_{end}\).

У випадку з соціальною мережею Facebook, вузлами можуть бути її користувачі. Між двома користувачами буде ребро, якщо вони друзі в Facebook.

\begin{center}
    \begin{tikzpicture}
    \footnotesize

    \begin{scope}[every node/.style={basic}]
    \node at (-1,1)   (alice)    {Alice};
    \node at (1,3)    (bob)      {Bob};
    \node at (3,0.5)  (charlie)  {Charlie};
    \node at (6,2)    (dave)     {Dave};
    \node at (9,2)    (eve)      {Eve};
    \node at (6,-1)   (fred)     {Fred};
    \node at (9,0)    (george)   {George};

    \tikzset{undirected/.style={draw,thick}}

    \path (alice)    edge [undirected]   (bob);
    \path (bob)      edge [undirected]   (charlie);
    \path (charlie)  edge [undirected]   (alice);
    \path (bob)      edge [undirected]   (dave);
    \path (charlie)  edge [undirected]   (dave);
    \path (dave)     edge [undirected]   (eve);
    \path (fred)     edge [undirected]   (george);
    \end{scope}

    \end{tikzpicture}
\end{center}

Оскільки ребра не мають напряму (відношення дружби у Facebook є симетричним), такий граф називається \textbf{неорієнтованим} (undirected).

Якщо ж ми розглянемо іншу соцмережу --- Twitter, то відношення Follow у ній несиметричне, і ребра Twitter-графу будуть напрямленими. Такий граф називається \textbf{орієнтованим} (directed), і його ребра іноді називають \textbf{дугами} (arcs):

\begin{center}
    \begin{tikzpicture}
    \footnotesize

    \begin{scope}[every node/.style={basic}]
    \node at (-1,1)   (alice)    {Alice};
    \node at (1,3)    (bob)      {Bob};
    \node at (3,0.5)  (charlie)  {Charlie};
    \node at (6,2)    (dave)     {Dave};
    \node at (9,2)    (eve)      {Eve};
    \node at (6,-1)   (fred)     {Fred};
    \node at (9,0)    (george)   {George};

    \tikzset{directed/.style={draw,->,thick,-{Latex[length=2.5mm]}}}

    \path (alice)    edge [directed]              (bob);
    \path (bob)      edge [directed, bend right]  (charlie);
    \path (charlie)  edge [directed]              (alice);
    \path (charlie)  edge [directed, bend right]  (bob);
    \path (bob)      edge [directed]              (dave);
    \path (charlie)  edge [directed]              (dave);
    \path (dave)     edge [directed, bend right]  (eve);
    \path (eve)      edge [directed, bend right]  (dave);
    \path (fred)     edge [directed]              (george);
    \path (fred)     edge [directed]              (dave);

    \end{scope}

    \end{tikzpicture}
\end{center}

Елементи графу також можуть мати числові ваги. Наприклад, якщо ми хочемо зобразити час поїздки між станціями київського метро, то отримаємо такий \textbf{зважений} (edge-weighted) граф:


\begin{center}
    \begin{tikzpicture}
    \footnotesize

    \tikzset{station/.style={draw, thick, text width=70pt, minimum height=30pt, text centered, rectangle, rounded corners=4pt, align=center, anchor=mid}}

    \begin{scope}[every node/.style={station}]
    \node at (-4,4)       (lukianivska)             {Лук’янівська};
    \node at (-3,0.85)    (zoloti-vorota)           {Золоті Ворота};
    \node at (0.75,0.85)  (teatralna)               {Театральна};
    \node at (-1,-3)      (ploshcha-lva-tolstoho)   {Площа Льва Толстого};
    \node at (3.5,-3.2)   (palats-sportu)           {Палац спорту};
    \node at (5,4)        (poshtova-ploshcha)       {Поштова площа};
    \node at (5,1.75)     (maidan-nezalezhnosti)    {Майдан Незалежності};
    \node at (6,-1)       (khreshchatyk)            {Хрещатик};
    \node at (9,-3.5)     (arsenalna)               {Арсенальна};
    \end{scope}

    \tikzset{undirected/.style={draw, thick, inner sep=5pt}}

    \path (lukianivska)             edge [undirected]  node [right] {5}   (zoloti-vorota);
    \path (zoloti-vorota)           edge [undirected]  node [above] {3}   (palats-sportu);
    \path (zoloti-vorota)           edge [undirected]  node [above] {1}   (teatralna);
    \path (ploshcha-lva-tolstoho)   edge [undirected]  node [above] {1}   (palats-sportu);
    \path (ploshcha-lva-tolstoho)   edge [undirected]  node [above] {5}   (maidan-nezalezhnosti);
    \path (maidan-nezalezhnosti)    edge [undirected]  node [right] {3}   (poshtova-ploshcha);
    \path (maidan-nezalezhnosti)    edge [undirected]  node [right] {1}   (khreshchatyk);
    \path (khreshchatyk)            edge [undirected]  node [below] {2}   (teatralna);
    \path (khreshchatyk)            edge [undirected]  node [above] {4}   (arsenalna);

    \end{tikzpicture}
\end{center}

Інколи граф може мати кілька ребер між однією й тією ж парою вершин (\textbf{кратні ребра}), або ж ребра, які з’єднують вершину саму з собою (\textbf{петлі}). Графи з такими ребрами називаються \textbf{мультиграфами}. Наприклад, уявімо граф поштової переписки:

\begin{center}
    \begin{tikzpicture}
    \footnotesize

    \tikzset{station/.style={draw, thick, text width=70pt, minimum height=20pt, text centered, rectangle, rounded corners=4pt, align=center, anchor=mid}}

    \begin{scope}[every node/.style={station}]
    \node at (-2,1)   (alice)    {Alice};
    \node at (1,3)    (bob)      {Bob};
    \node at (5,0.5)  (charlie)  {Charlie};
    \node at (8,2)    (dave)     {Dave};
    \end{scope}

    \tikzset{directed/.style={draw, ->, thick, inner sep=10pt, -{Latex[length=2.5mm]}}}

    \path (alice)      edge [directed]                                 node [left]                  {email 1}   (bob);
    \path (alice)      edge [directed, out=300, in=240, looseness=10]  node [below, inner sep=5pt]  {email 2}   (alice);
    \path (bob)        edge [directed, bend left]                      node [right]                 {email 3}   (charlie);
    \path (bob)        edge [directed]                                 node [above]                 {email 4}   (charlie);
    \path (bob)        edge [directed, bend right]                     node [left]                  {email 5}   (charlie);
    \path (charlie)    edge [directed]                                 node [right]                 {email 6}   (dave);

    \end{tikzpicture}
\end{center}



\section{Представлення графів}

Графи легко зобразити графічно, проте для того, щоб ефективно виконувати алгоритми на графах, ми повинні придумати спосіб представляти їх у коді та в пам’яті комп’ютера.

Одним з популярних представлень графів є \textbf{матриця суміжності} (adjacency matrix). Вона передбачає, що ми створюємо матрицю \(V \times V\), де рядками є початкові вершини, а стовпцями --- кінцеві вершини. Якщо з вершини \(v_i\) до вершини \(v_j\) є ребро, тоді відповідна клітинка \(A_{ij} = 1\). У випадку зваженого графа, замість 1 буде стояти вага ребра \(w_{(v_i, v_j)}\). Якщо граф неорієнтований, тоді матриця буде симетричною відносно основної діагоналі: \(A_{ij} = A_{ji}\).

\begin{center}
    \begin{tikzpicture}
    \footnotesize

    \begin{scope}[every node/.style={basic}]
    \node at (-1,1)   (alice)    {Alice};
    \node at (1,3)    (bob)      {Bob};
    \node at (3,0.5)  (charlie)  {Charlie};
    \node at (6,2)    (dave)     {Dave};
    \node at (9,2)    (eve)      {Eve};
    \node at (6,-1)   (fred)     {Fred};
    \node at (9,0)    (george)   {George};

    \tikzset{directed/.style={draw,->,thick,-{Latex[length=2.5mm]}}}

    \path (alice)    edge [directed]              (bob);
    \path (bob)      edge [directed, bend right]  (charlie);
    \path (charlie)  edge [directed]              (alice);
    \path (charlie)  edge [directed, bend right]  (bob);
    \path (bob)      edge [directed]              (dave);
    \path (charlie)  edge [directed]              (dave);
    \path (dave)     edge [directed, bend right]  (eve);
    \path (eve)      edge [directed, bend right]  (dave);
    \path (fred)     edge [directed]              (george);
    \path (fred)     edge [directed]              (dave);

    \end{scope}

    \end{tikzpicture}
\end{center}

\renewcommand{\arraystretch}{1.5}

\begin{center}
    \begin{tabular}{\adjmatrixlayout}
        \hline
        \textbf{From\textbackslash To} & Alice   & Bob   & Charlie   & Dave   & Eve   & Fred   & George \\ \hline
        Alice                          & 0       & 1     & 0         & 0      & 0     & 0      & 0 \\ \hline
        Bob                            & 0       & 0     & 1         & 1      & 0     & 0      & 0 \\ \hline
        Charlie                        & 1       & 1     & 0         & 1      & 0     & 0      & 0 \\ \hline
        Dave                           & 0       & 0     & 0         & 0      & 1     & 0      & 0 \\ \hline
        Eve                            & 0       & 0     & 0         & 1      & 0     & 0      & 0 \\ \hline
        Fred                           & 0       & 0     & 0         & 1      & 0     & 0      & 1 \\ \hline
        George                         & 0       & 0     & 0         & 0      & 0     & 0      & 0 \\ \hline
    \end{tabular}
\end{center}

Матрицю суміжності легко побудувати (якщо присвоїти кожній вершині числовий індекс). Також вона швидко (за \(O(1)\)) може дати відповідь на питання «Чи існує ребро між вершинами \(v_i\) та \(v_j\)?»

Проте, представлення графів матрицею суміжності має такі недоліки:
\begin{itemize}
    \item Нам завжди потрібно створювати матрицю розміром \(V \times V\). Це означає, що нам потрібно затрачати квадратичну кількість пам’яті відносно кількості вершин.
    \item Часто нам потрібно перелічити всіх сусідів вершини \(v_i\). З матрицею суміжності, ми мусимо просканувати весь \(i\)-й рядок, щоб дати відповідь на це питання.
    \item Матриця суміжності не може зберігати кратні ребра.
\end{itemize}

Інше поширене представлення --- \textbf{список суміжності} (adjacency list), при якому для кожної вершини вказується список ребер, які з неї виходять (або ж просто список суміжних вершин). Таким чином, ми використовуємо рівно стільки пам’яті, скільки у нас є елементів графу, а також можемо швидко отримати усі суміжні вершини.

\begin{align*}
    Alice:    & \  [Edge(Alice, Bob)] \\
    Bob:      & \  [Edge(Bob, Charlie), Edge(Bob, Dave)] \\
    Charlie:  & \  [Edge(Charlie, Alice), Edge(Charlie, Bob), Edge(Charlie, Dave)] \\
    Dave:     & \  [Edge(Dave, Eve)] \\
    Eve:      & \  [Edge(Eve, Dave)] \\
    Fred:     & \  [Edge(Fred, Dave), Edge(Fred, George)] \\
    George:   & \  []
\end{align*}

В коді ми можемо описати це так:

\lstinputlisting{code/graph_repr_edge_list.py}

\end{document}