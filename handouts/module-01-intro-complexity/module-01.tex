\documentclass[12pt,a4paper]{report}

% ----------------------------
% Enable multilingual content
% ----------------------------
\usepackage[T1, T2A]{fontenc}
\usepackage[utf8]{inputenc}
\usepackage[ukrainian, english]{babel}

% -------------------------------
% Custom formatting for headings
% -------------------------------
\usepackage{titlesec}

% Chapters
\titleformat
    {\chapter}                      % command
    [display]                       % shape
    {\large\bfseries}               % format
    {\vspace{-3em} Модуль \thechapter}            % label
    {0.5ex}                         % sep
    {
        \rule{\textwidth}{1.5pt}
        \vspace{0.5ex}
        \centering
        \mdseries
        \itshape
    }                               % before-code
    [
        \vspace{-1.7ex}
        \rule{\textwidth}{0.3pt}
    ]                               % after-code

% Sections
\titleformat
    {\section}                      % command
    {\normalsize\bfseries}          % format
    {\thesection}                   % label
    {1em}                           % sep
    {}                              % before-code
    []                              % after-code

% Subsections
\titleformat
    {\subsection}                   % command
    {\normalsize\bfseries\itshape}  % format
    {\thesubsection}                % label
    {1em}                           % sep
    {}                              % before-code
    []                              % after-code

% ----------------
% List formatting
% ----------------
\usepackage{paralist}
    \let\itemize\compactitem
    \let\enditemize\endcompactitem
    \let\enumerate\compactenum
    \let\endenumerate\endcompactenum
    \let\description\compactdesc
    \let\enddescription\endcompactdesc
    \pltopsep=\medskipamount
    \plitemsep=1pt
    \plparsep=1pt

% ---------------------------------------------------------
% Commands that can be switched on for debugging purposes.
% ---------------------------------------------------------

% Display document layout guides.
%\usepackage[showframe]{geometry}

% Enable this and use \layout in the document to include a special page with document layout metrics.
%\usepackage{layout}


\begin{document}

\chapter{Вступ. Аналіз складності алгоритмів.}

\section{Що таке алгоритм?}
\emph{Алгоритм} --- це покроковий набір інструкцій для вирішення конкретної задачі за скінченний час та з використанням скінченного простору (пам'яті).

\vspace{1.5em}
Існує кілька способів вираження алгоритму:
\begin{itemize}
    \item Словесно --- наприклад, реченнями української чи англійської мови.
    \item Графічно --- наприклад, блок-схемою чи іншими формалізованими діаграмами.
    \item Мовою програмування.
\end{itemize}

\subsection*{Опис задачі}
Навіть найменші деталі можуть вплинути на вибір алгоритму для вирішення задачі. Тому при її описі варто вказувати:
\begin{enumerate}
    \item Формат вхідних даних.

        {\itshape Рядок з \(N\)\ чисел, числа розділяються одним пробілом.}

    \item Тип та діапазон значень вхідних даних.

        {\itshape Чисел може бути від 1 до 100. Всі числа --- натуральні, не більші за 50000.}

    \item Що потрібно обчислити.

        {\itshape Знайти мінімальний елемент \(N_{min}\) з-поміж заданих \(N\) чисел.}

    \item Формат вихідних даних.

        {\itshape Одне натуральне число --- \(N_{min}\).}

\end{enumerate}

\subsection*{Аналіз алгоритмів}
Три запитання, які дозволяють проаналізувати алгоритм:
\begin{enumerate}
    \item Чи алгоритм правильний?
    \item Наскільки алгоритм хороший?
    \item Чи можна вирішити задачу краще?
\end{enumerate}

\section{Чи алгоритм правильний?}

\section{Наскільки алгоритм хороший?}

\section{Чи можна вирішити задачу краще?}

\end{document}
