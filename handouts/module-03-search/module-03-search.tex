\documentclass[12pt,a4paper]{report}

\usepackage{amsmath}
\usepackage{tikz}

\usetikzlibrary{arrows.meta}
\usetikzlibrary{shapes.misc}

% ----------------------------
% Enable multilingual content
% ----------------------------
\usepackage[T1, T2A]{fontenc}
\usepackage[utf8]{inputenc}
\usepackage[ukrainian, english]{babel}

% ---------------------
% Page layout settings
% ---------------------

% Set showframe to true to display layout guides.
\usepackage[showframe=false]{geometry}

% Enable this and use \layout in the document to include a special page with document layout metrics.
%\usepackage{layout}

% Set page margins
\geometry{margin=1in}

% ---------------------------
% Custom formatting for code
% ---------------------------

\usepackage{listings}
\usepackage{color}
\usepackage{inconsolata}
\usepackage{upquote}

\definecolor{commentcolor}{rgb}{0.6,0.6,0.6}


% Style to be used by all Python source listings.
\lstdefinestyle{codePythonStyle}{
    language=Python,
    commentstyle=\color{commentcolor},
    basewidth=0.53em,
    stepnumber=1,
    numbersep=8pt,
    tabsize=4,
    showspaces=false,
    showstringspaces=false
}

% A shorthand syntax for language-neutral inline code blocks, e.g. |foo|
\lstMakeShortInline[language=]|

% A shorthand syntax for language-neutral inline code blocks, e.g. \code{foo}
\newcommand{\code}{\lstinline[language=]}

% Language-neutral multiline code blocks.
\lstnewenvironment{codeblock}
    {\lstset{language=}}
    {}


% Use Python by default.
\lstset{
    basicstyle=\fontencoding{T1}\selectfont\ttfamily\small,
    style=codePythonStyle
}

\usepackage[unicode]{hyperref}

% Colors for hyperlinks
\hypersetup{%
  colorlinks=false,            % hyperlinks will be black
  linkbordercolor={0.3 0.4 1}, % hyperlink borders will be blue
  pdfborderstyle={/S/U/W 1}    % border style will be underline of width 1pt
}

% Avoid header in the TOC
\makeatletter
\renewcommand\tableofcontents{%
    \@starttoc{toc}%
}
\makeatother

% Make hyperlinks underlined instead of bounded by a rectangle.
\makeatletter
\Hy@AtBeginDocument{%
  \def\@pdfborder{0 0 1}            % Overrides border definition set with colorlinks=true
  \def\@pdfborderstyle{/S/U/W 1}    % Overrides border style set with colorlinks=true
                                    % Hyperlink border style will be underline of width 1pt
}
\makeatother

% -------------------------------
% Custom formatting for headings
% -------------------------------
\usepackage{titlesec}

% Chapters
\titleformat
    {\chapter}                      % command
    [display]                       % shape
    {\large\bfseries}               % format
    {\vspace{-3em} Модуль \thechapter}            % label
    {0.5ex}                         % sep
    {
        \rule{\textwidth}{1.5pt}
        \vspace{0.5ex}
        \centering
        \mdseries
        \itshape
    }                               % before-code
    [
        \vspace{-1.7ex}
        \rule{\textwidth}{0.3pt}
    ]                               % after-code

% Sections
\titleformat
    {\section}                      % command
    {\normalsize\bfseries}          % format
    {\thesection}                   % label
    {1em}                           % sep
    {}                              % before-code
    []                              % after-code

% Subsections
\titleformat
    {\subsection}                   % command
    {\normalsize\bfseries\itshape}  % format
    {\thesubsection}                % label
    {1em}                           % sep
    {}                              % before-code
    []                              % after-code

% ----------------
% List formatting
% ----------------
\usepackage{paralist}
    \let\itemize\compactitem
    \let\enditemize\endcompactitem
    \let\enumerate\compactenum
    \let\endenumerate\endcompactenum
    \let\description\compactdesc
    \let\enddescription\endcompactdesc
    \pltopsep=\medskipamount
    \plitemsep=1pt
    \plparsep=1pt

% ---------------------------------
% Custom formatting for paragraphs
% ---------------------------------
\setlength{\parskip}{0.5em}
\setlength{\parindent}{0em}

\newenvironment{pagebottomtext}{\begin{center}\par\vspace*{\fill}}{\clearpage\end{center}}




% TODO: Remove this after merging the modules into a single master file.
\setcounter{chapter}{2}

\tikzset{basic/.style = {draw, text width=50pt, rectangle, align=center}}
\tikzset{treenode/.style={draw, thick, text width=50pt, minimum height=20pt, text centered, rectangle, rounded corners=4pt, align=center, anchor=mid}}


\begin{document}

\chapter{Пошук}

% Insert table of contents.
\begingroup
\let\clearpage\relax
\tableofcontents
\endgroup



\section{Задача пошуку}

Пошук дозволяє нам справлятися з величезними потоками інформації у повсякденному житті, відокремлюючи тільки потрібні для нас дані. Щоразу, коли ми шукаємо інформацію в Google, отримуємо телефонний дзвінок від людини, яка є у нашому списку контактів, чи вибираємо купюри з гаманця, щоб оплатити якийсь товар готівкою, виконується задача пошуку.

За аналогією з попередніми задачами, спробуємо описати задачу пошуку більш формально.

У нас є \(N\) об’єктів, кожен з яких має певний унікальний ключовий атрибут \(key\). Ми хочемо зберегти усі ці об’єкти так, щоб кожного із них можна було потім знайти за ключем \(key\) (або виявити, що такого об’єкта не існує). Наприклад, якщо в телефонному програмному забезпеченні потрібно показати, яка людина нам телефонує, маючи номер її телефону, то в телефонній книзі об’єктами будуть люди, а ключовим атрибутом буде номер телефону.

Структура даних, яка реалізовує задачу пошуку, повинна підтримувати такі базові операції:

\begin{enumerate}
    \item |Add(key, value)| -- додати елемент |value| із ключем |key| до бази пошуку.
    \item |Get(key)| -- знайти в базі eлемент (|value|) за ключем |key|, або повідомити, що такого елемента нема.
\end{enumerate}

Деякі структури можуть підтримувати додаткові операції, наприклад:
\begin{itemize}
    \item |Remove(key)| -- вилучити елемент з бази.
    \item |Exists(key)| -- перевірити, чи елемент існує в базі.
    \item |GetAllInRange(rangeStart, rangeEnd)| -- знайти усі елементи, де ключ лежить в діапазоні від |rangeStart| до |rangeEnd|.
\end{itemize}



\section {Бінарний пошук}

Припустимо, нам потрібно зберігати базу користувачів веб-сайту, де кожен користувач має унікальний ідентифікатор -- |username|. Нехай у нас є користувачі |alice|, |bob|, |charlie|, |dave|, |eve|, |fred|, |george|, тощо.

Спробуємо реалізувати структуру даних для пошуку. Для початку, використаємо звичайний невпорядкований масив. Тоді операція |Add| буде дописувати елементи в кінець масиву, а операція |Get| -- сканувати весь масив, доки не знайде в ньому елемент з заданим ключем.

Після додавання кількох користувачів в довільному порядку наш масив може виглядати так:

\begin{center}
    \begin{tikzpicture}
        \footnotesize
        \draw[step=3cm,gray,very thin] (0,0) grid (15,1);
        \draw node at (11.5,2.0) (insert-label) {місце вставки нового елемента};

        \normalsize
        \draw node at (1.5,0.5) {dave};
        \draw node at (4.5,0.5) {alice};
        \draw node at (7.5,0.5) {fred};
        \draw node at (10.5,0.5) {bob};
        \draw node at (13.5,0.5) (insert-cell) {};

        \draw [draw,->,thick,-{Latex[length=2.5mm]}] (insert-label) -- (insert-cell);
    \end{tikzpicture}
\end{center}

Як і у випадку з Priority Queue з попереднього модуля, швидкодія такого підходу доволі очевидна: ми будемо вставляти новий елемент за \(O(1)\), проте час пошуку буде лише \(O(N)\). Якщо припустити, що ми змінюємо базу даних значно рідше, ніж шукаємо в ній елементи, такий підхід буде надто повільним.

Проте, ми вже володіємо прийомами сортування. Можливо, вигідніше буде впорядкувати дані, щоб потім пошук по них відбувався швидше?

\begin{center}
    \begin{tikzpicture}
        \footnotesize
        \draw[step=2cm,gray,very thin] (0,0) grid (16,1);
        \normalsize
        \draw node at (1,0.5) {alice};
        \draw node at (3,0.5) {bob};
        \draw node at (5,0.5) {charlie};
        \draw node at (7,0.5) {dave};
        \draw node at (9,0.5) {eve};
        \draw node at (11,0.5) {fred};
        \draw node at (13,0.5) {george};
        \draw node at (15,0.5) {henry};
    \end{tikzpicture}
\end{center}

Припустимо, ми шукаємо користувача |bob|. Тепер глянемо на довільний елемент нашого впорядкованого масиву. Нехай це буде |fred|. Оскільки |bob| < |fred|, ми можемо бути впевнені, що |bob| не може бути правіше, ніж |fred| у масиві. Тому ми можемо взагалі відкинути всю праву частину, і продовжувати пошук |bob| лише в лівій!

На цій логіці базується алгоритм \textbf{бінарного пошуку}:
\begin{enumerate}
    \item Якщо масив порожній --- шуканого елемента не існує.
    \item Інакше --- порівнюємо середній елемент з шуканим значенням.
    \item Якщо середній елемент дорівнює шуканому значенню, повертаємо цей елемент як результат пошуку.
    \item Якщо середній елемент більший за шукане значення, повторюємо кроки 1--5 для лівої половини масиву.
    \item Якщо середній елемент менший за шукане значення, повторюємо кроки 1--5 для правої половини масиву.
\end{enumerate}

Для нашого прикладу з користувачами, спробуємо знайти |charlie|:
\begin{enumerate}
    \item Середній елемент --- |dave|. |charlie| < |dave|, тому звужуємо пошук до лівої половини (|alice| -- |charlie|).
    \item Тепер середній елемент --- |bob|. |charlie| > |bob|, тому звужуємо пошук до правої половини (|charlie| -- |charlie|).
    \item Тепер середній елемент --- |charlie|. Ми знайшли шуканий елемент.
\end{enumerate}

Логіка бінарного пошуку є рекурсивною, проте в коді ми можемо замінити рекурсію на цикл:

\lstinputlisting{code/binary_search.py}

Таким чином, на кожному кроці наша область пошуку скорочується вдвічі, тому ми можемо знайти будь-який елемент за \(O(\log N)\) часу. Ми лише мусимо затратити певний час на те, щоб спочатку впорядкувати дані. Тому бінарний пошук --- це хороший алгоритм, коли нам потрібно багато разів шукати елементи у незмінному наборі даних.

Зрозуміло, що ми все ще маємо проблему із тим, що вставляти дані в посортований масив ми можемо лише за \(O(N)\). Про можливі її рішення ми поговоримо в наступних розділах.

Інколи у нас може бути потреба серед усіх рівних елементів масиву знайти найлівіший чи найправіший:

\begin{center}
    \begin{tikzpicture}
        \footnotesize
        \draw[step=1cm,gray,very thin] (0,0) grid (10,1);
        \draw node at (1.5,2.0) (leftmost-label) {найлівіший елемент 3};
        \draw node at (6.5,2.0) (rightmost-label) {найправіший елемент 3};

        \normalsize
        \draw node at (0.5,0.5) {0};
        \draw node at (1.5,0.5) {1};
        \draw node at (2.5,0.5) (leftmost-cell) {3};
        \draw node at (3.5,0.5) {3};
        \draw node at (4.5,0.5) {3};
        \draw node at (5.5,0.5) (rightmost-cell) {3};
        \draw node at (6.5,0.5) {4};
        \draw node at (7.5,0.5) {5};
        \draw node at (8.5,0.5) {6};
        \draw node at (9.5,0.5) {9};

        \draw [draw,->,thick,-{Latex[length=2.5mm]}] (leftmost-label) -- (leftmost-cell);
        \draw [draw,->,thick,-{Latex[length=2.5mm]}] (rightmost-label) -- (rightmost-cell);

    \end{tikzpicture}
\end{center}

Цю задачу все ще можна вирішити бінарним пошуком за \(O(\log N)\) --- все тільки залежить від тонких деталей, як саме ми будемо порівнювати елементи та скорочувати діапазон пошуку:

\lstinputlisting{code/binary_search_edge.py}



\section{Бінарні дерева пошуку}

У попередньому модулі ми уже зустрічали частковий випадок бінарного дерева --- бінарну піраміду (Binary Heap), яка застосовувалася для пошуку максимумів (мінімумів) у змінному потоці даних.

Повернемося до нашого прикладу з пошуком користувачів у впорядкованому масиві. Спробуємо проаналізувати, за якою логікою ми порівнюємо середні елементи при бінарному пошуку.

Спочатку ми порівнюємо з |dave|. Якщо те, що ми шукаємо, менше за |dave|, ми порівнюємо з |bob|, якщо більше --- з |fred|. Якщо те, що ми шукаємо, більше за |fred|, ми порівнюємо з |george|, і так далі. Таким чином, логіку порівнянь можна описати таким бінарним деревом:

\begin{center}
    \begin{tikzpicture}

    \begin{scope}[every node/.style={treenode}]
    \node                                                  (level1-1) {dave};
    \node [below of = level1-1,xshift=-120pt,yshift=-20pt] (level2-1) {bob};
    \node [below of = level1-1,xshift= 120pt,yshift=-20pt] (level2-2) {fred};
    \node [below of = level2-1,xshift=-60pt,yshift=-20pt]  (level3-1) {alice};
    \node [below of = level2-1,xshift= 60pt,yshift=-20pt]  (level3-2) {charlie};
    \node [below of = level2-2,xshift=-60pt,yshift=-20pt]  (level3-3) {eve};
    \node [below of = level2-2,xshift= 60pt,yshift=-20pt]  (level3-4) {george};
    \node [below of = level3-4,xshift= 30pt,yshift=-20pt]  (level4-8) {henry};

    \path [draw,thick] (level4-8) -- (level3-4);
    \path [draw,thick] (level3-4) -- (level2-2);
    \path [draw,thick] (level3-3) -- (level2-2);
    \path [draw,thick] (level3-2) -- (level2-1);
    \path [draw,thick] (level3-1) -- (level2-1);
    \path [draw,thick] (level2-2) -- (level1-1);
    \path [draw,thick] (level2-1) -- (level1-1);
    \end{scope}

    \draw node at (-2.2,-0.4) {<};
    \draw node at ( 2.2,-0.4) {>};
    \draw node at (-5.5,-2.3) {<};
    \draw node at ( 5.5,-2.3) {>};
    \draw node at (-2.8,-2.3) {>};
    \draw node at ( 2.8,-2.3) {<};
    \draw node at ( 7.2,-4.2) {>};

    \end{tikzpicture}
\end{center}

Таке дерево називається \textbf{бінарним деревом пошуку} (Binary Search Tree, BST). Це підвид бінарного дерева, який має додаткову властивість: для будь-якого вузла всі елементи в лівому піддереві менші за вузол, в правому --- більші за вузол.

Зберігання даних у вигляді дерева має певні вигоди: ми використовуємо рівно стільки пам’яті, скільки у нас є даних, не резервуючи вільних комірок, як би це було у випадку з масивом; а також, ми можемо швидше вставляти нові елементи, оскільки нам не потрібно зсовувати всі наступні елементи, щоб додати елемент всередину.

Якби нам потрібно було вставити елемент |carl|, ми можемо достатньо швидко підшукати для нього нове місце:

\begin{center}
    \begin{tikzpicture}
    \begin{scope}[every node/.style={treenode}]
    \node                                                              (level1-1) {dave};
    \node [below of = level1-1,xshift=-120pt,yshift=-20pt]             (level2-1) {bob};
    \node [below of = level1-1,xshift= 120pt,yshift=-20pt]             (level2-2) {fred};
    \node [below of = level2-1,xshift=-60pt,yshift=-20pt]              (level3-1) {alice};
    \node [below of = level2-1,xshift= 60pt,yshift=-20pt]              (level3-2) {charlie};
    \node [below of = level2-2,xshift=-60pt,yshift=-20pt]              (level3-3) {eve};
    \node [below of = level2-2,xshift= 60pt,yshift=-20pt]              (level3-4) {george};
    \node [below of = level3-2,xshift=-30pt,yshift=-20pt,color=cyan]   (level4-3) {carl};    
    \node [below of = level3-4,xshift= 30pt,yshift=-20pt]              (level4-8) {henry};

    \path [draw,thick,color=cyan] (level4-3) -- (level3-2);
    \path [draw,thick] (level4-8) -- (level3-4);
    \path [draw,thick] (level3-4) -- (level2-2);
    \path [draw,thick] (level3-3) -- (level2-2);
    \path [draw,thick] (level3-2) -- (level2-1);
    \path [draw,thick] (level3-1) -- (level2-1);
    \path [draw,thick] (level2-2) -- (level1-1);
    \path [draw,thick] (level2-1) -- (level1-1);
    \end{scope}

    \end{tikzpicture}
\end{center}

Виглядає, що якщо ми будемо дотримуватися властивості бінарного дерева пошуку, ми можемо завжди вставити та знайти будь-який елемент за \(O(\log N)\) часу. Проте існують випадки, коли бінарне дерево пошуку виявиться дуже неефективною структурою.

Уявимо собі, що елементи надходять до нашої бази в такому порядку: |alice|, |bob|, |charlie|, |dave|. Якщо ми будемо по черзі вставляти їх у бінарне дерево пошуку, ми отримаємо таке:

|Add("alice", new Person())|

\begin{center}
    \begin{tikzpicture}
    \begin{scope}[every node/.style={treenode}]
    \node [color=cyan] (level1-1) {alice};
    \end{scope}

    \end{tikzpicture}
\end{center}

|Add("bob", new Person())|

\begin{center}
    \begin{tikzpicture}
    \begin{scope}[every node/.style={treenode}]
    \node                                                            (level1-1) {alice};
    \node [below of = level1-1,xshift=60pt,yshift=-10pt, color=cyan] (level2-1) {bob};
    \end{scope}

    \path [draw,thick,color=cyan] (level2-1) -- (level1-1);
    \end{tikzpicture}
\end{center}

|Add("charlie", new Person())|

\begin{center}
    \begin{tikzpicture}
    \begin{scope}[every node/.style={treenode}]
    \node                                                           (level1-1) {alice};
    \node [below of = level1-1,xshift=60pt,yshift=-10pt]            (level2-1) {bob};
    \node [below of = level2-1,xshift=60pt,yshift=-10pt,color=cyan] (level3-1) {charlie};
    \end{scope}

    \path [draw,thick]            (level2-1) -- (level1-1);
    \path [draw,thick,color=cyan] (level3-1) -- (level2-1);
    \end{tikzpicture}
\end{center}

|Add("dave", new Person())|

\begin{center}
    \begin{tikzpicture}
    \begin{scope}[every node/.style={treenode}]
    \node                                                           (level1-1) {alice};
    \node [below of = level1-1,xshift=60pt,yshift=-10pt]            (level2-1) {bob};
    \node [below of = level2-1,xshift=60pt,yshift=-10pt]            (level3-1) {charlie};
    \node [below of = level3-1,xshift=60pt,yshift=-10pt,color=cyan] (level4-1) {dave};
    \end{scope}

    \path [draw,thick]            (level2-1) -- (level1-1);
    \path [draw,thick]            (level3-1) -- (level2-1);
    \path [draw,thick,color=cyan] (level4-1) -- (level3-1);
    \end{tikzpicture}
\end{center}

Хоч ми отримали цілком коректне бінарне дерево пошуку з точки зору властивостей, в цьому випадку ми можемо витрачати аж \(O(N)\) часу для того, щоб вставити або знайти елемент. Виглядає, що ми можемо досягнути хорошої швидкодії лише тоді, коли ліве та праве піддерево будуть врівноважені за висотою, і було б чудово, якби дерево могло врівноважити себе самостійно при вставці та вилученні елементів.

\begin{center}
    \begin{tikzpicture}
    \begin{scope}[every node/.style={treenode}]
    \node                                                  (level1-1) {charlie};
    \node [below of = level1-1,xshift=-120pt,yshift=-20pt] (level2-1) {bob};
    \node [below of = level1-1,xshift= 120pt,yshift=-20pt] (level2-2) {dave};
    \node [below of = level2-1,xshift=-60pt,yshift=-20pt]  (level3-1) {alice};

    \path [draw,thick] (level3-1) -- (level2-1);
    \path [draw,thick] (level2-2) -- (level1-1);
    \path [draw,thick] (level2-1) -- (level1-1);
    \end{scope}

    \end{tikzpicture}
\end{center}

Такі дерева існують, і вони називаються \textbf{збалансованими бінарними деревами пошуку} (Balanced Binary Search Tree). Вони мають ті ж самі властивості, що й бінарні дерева пошуку, проте при зміні елементів вони проводять різноманітні операції обміну вузлів, так щоб урівноважити ліве та праве піддерево. Це дозволяє їм забезпечити гарантовану швидкодію \(O(\log N)\) для вставки, пошуку та вилучення елементів. Найвідомішими такими деревами є AVL-дерево (AVL Tree) та червоно-чорне дерево (Red-Black Tree), про логіку роботи яких можна прочитати в матеріалах для поглибленого вивчення, вказаних в кінці модуля.



\section{Хеш-таблиці}

Коли ми користуємося масивами в коді, ми маємо чудову гарантію, що доступ до будь-якого елемента за його індексом відбудеться за \(O(1)\) часу. Проте, індекси у масиві --- цілочисельні, а нам часто хотілося б мати масив з індексами, які можуть бути стрічками, датами, довільними об’єктами тощо. Наприклад, якби ми хотіли шукати день народження за прізвищем людини: |birthday = birthdays["Alice"]|. Іншими словами, нам потрібно реалізувати \textbf{асоціативний масив} (associative array).

Як ми бачили з попередніх розділів, таку функціональність ми можемо реалізувати за допомогою збалансованих бінарних дерев пошуку, і мати швидкодію \(O(\log N)\) для кожної з основних операцій.

Чи можемо ми досягти кращої швидкодії? Якби ми могли перетворити стрічкові ключі в цілочисельні, ми б звели задачу до використання звичайного масиву, і отримали б бажане \(O(1)\). Проте проблема в тому, що стрічок у світі існує нескінченна кількість, тому якби ми спробували присвоїти кожній можливій стрічці числовий індекс, нам би довелося виділяти дуже великі масиви. Скоріш за все, пам’яті у нас буде надто мало, навіть щоб підтримати усі можливі 5-символьні стрічки, не кажучи вже про 50-символьні.

Цю проблему допомагає вирішити \textbf{хешування}. Замість того, щоб виділяти масив на усі можливі стрічки в світі, ми виділяємо невеликий масив (наприклад, розміром в 23 елементи), і вводимо функцію, яка довільну стрічку відображає в число \([0..22]\). Така функція, що відображає вхідні дані довільного розміру в множину фіксованого розміру, називається \textbf{хеш-функцією}. Наприклад, ми б могли реалізувати цю функцію таким чином: беремо перші 3 символи стрічки, обчислюємо добуток їх ASCII-кодів, і беремо остачу від його ділення на 23:

\begin{lstlisting}
    def hash(s):
        return (ord(s[0]) * ord(s[1]) * ord(s[2])) % 23
\end{lstlisting}

Тепер ми можемо виділити масив на 23 елементи, і хеш-функція буде підказувати нам, до якої комірки нам звертатися з даним ключем:

\begin{lstlisting}
    array = [None] * 23

    def get(key):
        return array[hash(key)]

    def add(key, value):
        array[hash(key)] = value

    add("Alice", "1990-01-01")
    add("Bob", "1990-12-20")
    print get("Alice")
\end{lstlisting}

На жаль, у такого підходу є слабка сторона. Оскільки хеш-функція відображає нескінченну множину у скінченну, неодмінно знайдуться такі два значення \(a, b\), де \(a \neq b\), але \(hash(a) = hash(b)\). Це явище називається \textbf{колізією хеш-функції}. Наприклад, для нашої хеш-функції вище колізію спричинять ключі |Alice| та |Alistair|. Ми мусимо вирішити, що ми будемо робити з колізіями, зберігаючи наші елементи в масиві (хеш-таблиці).

Одне з найпопулярніших рішень --- зберігати в кожній комірці масиву не один елемент, а зв’язаний список елементів:

\begin{center}
    \begin{tikzpicture}

    \begin{scope}[every node/.style={basic}]
    \node [text width=20pt, text height=14pt, text depth=6pt, text centered] at (0,0) (bucket-0) {0};
    \node [text width=20pt, text height=14pt, text depth=6pt, text centered] at (0,-1) (bucket-1) {1};
    \node [text width=20pt, text height=14pt, text depth=6pt, text centered] at (0,-2) (bucket-2) {2};
    \node [text width=20pt, text height=14pt, text depth=6pt, text centered] at (0,-3) (bucket-3) {3};
    \node [text width=20pt, text height=14pt, text depth=6pt, text centered] at (0,-4) (bucket-ellipsis) {...};
    \node [text width=20pt, text height=14pt, text depth=6pt, text centered] at (0,-5) (bucket-22) {22};

    \node [text width=70pt, text height=8pt, text depth=10pt, text centered] at (3,-3) (alice) {|Alice|};
    \node [text width=70pt, text height=8pt, text depth=10pt, text centered] at (6.5,-3) (alistair) {|Alistair|};
    \node [text width=70pt, text height=8pt, text depth=10pt, text centered] at (3,-1) (bob) {|Bob|};
    \end{scope}

    \draw node [text width=70pt, text height=8pt, text depth=10pt, text centered] at (3,-3.35) {|1990-01-01|};
    \draw node [text width=70pt, text height=8pt, text depth=10pt, text centered] at (6.5,-3.35) {|1992-05-10|};
    \draw node [text width=70pt, text height=8pt, text depth=10pt, text centered] at (3,-1.35) {|1990-12-20|};

    \path [draw,->,thick,-{Latex[length=2.5mm]}] (bucket-3) -- (alice);
    \path [draw,->,thick,-{Latex[length=2.5mm]}] (alice) -- (alistair);
    \path [draw,->,thick,-{Latex[length=2.5mm]}] (bucket-1) -- (bob);

    \end{tikzpicture}

\end{center}

В такому випадку, хеш-функція буде лише підказувати нам номер комірки, а далі ми будемо проходити зв’язаний список, порівнюючи кожен елемент з тим, що ми шукаємо. Якщо ми дійшли до кінця списку, елемента не існує.

Очевидно, що якщо хеш-функція буде мати нерівномірний розподіл значень, то певні комірки хеш-таблиці будуть переповненими, а деякі -- взагалі порожніми. Тому хороша хеш-функція --- це така, що рівномірно балансує вхідні дані по можливих комірках. Це дасть нам змогу очікувати від хеш-таблиці швидкодії \(O(1 + \frac{n}{k})\), де \(n\) --- кількість елементів, записаних у таблицю, \(k\) --- кількість комірок, які ми виділили (в нашому випадку \(k = 23\)).

Іншою стратегією роботи з колізіями є Linear Probing, яка у випадку колізії намагається записати елемент у сусідні незайняті комірки.



\section*{Ресурси для поглибленого вивчення}
\begin{enumerate}
    \item Robert Sedgewick, Kevin Wayne. Algorithms, 4\textsuperscript{th} edition: {\itshape Chapter 3: Searching}.
    \item Cormen, Leierson, Rivest, Stein. Introduction to Algorithms, 3\textsuperscript{rd} edition: {\itshape Chapter 11: Hash Tables}.
    \item Cormen, Leierson, Rivest, Stein. Introduction to Algorithms, 3\textsuperscript{rd} edition: {\itshape Chapter 12: Binary Search Trees}.
    \item Cormen, Leierson, Rivest, Stein. Introduction to Algorithms, 3\textsuperscript{rd} edition: {\itshape Chapter 13: Red-Black Trees}.
\end{enumerate}

\end{document}